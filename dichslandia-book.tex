\section{ДІЧСЛАНДІЯ:}

\subsection{12 днів в країні водоспадів, гейзерів, вулканів та інших
видумок
природи}
От і настав час описати, без перебільшення, наяскравішу подорож цього
року --- поїздку в Ісландію. Аби не писати лонгрід, спробую зібрати один
лаконічний пост із загальним, а також окремо про кожен день маршруту.
Надіюсь вам буде цікаво і корисно і зовсім згодом ви, так як ми свого
часу, почнете планувати поїздку.

Відразу скажу, що весь контент ми публікували по хештегу
\emph{\#дічсландія} у Твіттері та Інстаграмі та у меншій мірі у
Фейсбуці, так що можете поринути у події так, як вони були в реальному
часі. 

\subsection{Квитки}

Як і годиться, так і до цієї епічної подорожі ми готувалось дуже давно.
Тому квитки взяли ще в жовтні 2014 року  Нам повезло, так як брали ще
до стрибків курсу. Вирішили комбінувати лоукости із якимись дешевими
перельотами із України.

Вийшла така схема: на МАУ із Києва в Женеву, потім \emph{Easyjet} в
Рейк'явік. Оскільки ми знали, що будемо їхати із величезним багажем,
тому додатково відразу купили багажне місце в \emph{Easyjet}. Зараз є
дешевші варіанти, наприклад із Гданська прямим рейсом від \emph{Wizz
Air} (у жовтні 2014 його ще не було).

\begin{quote}
Сума, яка вийшла в загальному за квитки --- 250\euro{} з людини.
\end{quote}

\subsection{Проживання}

На «перевалочному» пункті у Женеві в загальному ми мали ночувати 3 ночі
(1 ніч по прильоті і 2 по відльоті в Україну), тому взяли найдешевший
готель --- \emph{Ibis Budget Hotel}, який не далеко від аеропорта. Житло
у Женеві дуже дороге, відповідно три дні в готелі обійшлись в 82\euro{}.

Жити у самій Ісландії ми збирались в наметах у кемпінгах, тому
забронювали проживання в гестхаусі лише на останній день у Рейк'явіку
(28\euro{}).

Для проживання в кемпінгах є непогана програма \emph{Iceland Camping
Card}, яка дозволяє жити 28 днів двом особам у мережі кемпінгів по всій
Ісландії (110\euro{}). Ми взяли дві картки і вирішили, що за 5 людину
будемо доплачувати безпосередньо на місці. Як ми взнали потім і на цьому
можна було б зекономити, адже за всі 10 ночей в нас перевірили картку
лише 3 чи 4 рази, а за 5 людину ми платили лише 2 рази, та й то, при
бажанні, можна було би скіпнути і нічого не платити. Але карма і все
таке, тому оплатити і бути спокійним 

\begin{quote}
Отже сума на житло з особи за увесь період подорожі -- 158\euro{}
\end{quote}

\subsection{Автомобіль}

Оренда автомобіля --- одна із найдорожчих витрат у цій поїздці. Ми
вирішили брати позашляховик, щоб проїхати по різних пересічних дорогах.
Вирішили взяти \emph{Ford Kuga} 4×4 з ручною коробкою передач. По
прильоті --- саме цієї машини не виявилось, тому нам дали Mazda CX5, яка
по класу така ж, щоправда із автоматичною коробкою. Хлопцям за кермом
було приємніше, на порядок.

Орендували тут -- http://rentalcars.com

\begin{quote}
Загальна кількість кілометрів за подорож -- 3200!
\end{quote}

\begin{quote}
Сума оренди автомобіля на 10 днів на людину: 152\euro{}
\end{quote}

Разом із орендованою карткою нам дали дисконт у мережі АЗС Olis, разом
із кемпінг-карткою нам дали дисконту картку в мережу АЗС Orkan. Перша
трошки дорожча в ціні працює по моделі --- залив бак-заплатив, друга
дешевше --- по препейду --- поставив гроші на картку --- користуєшся
поки не вичерпаєш. Переважно використовували Orkan. Коли заливали
останній раз перед здачею машини заправились в \emph{Olis}, щоб
випадково гроші не залишились на препейді.

\begin{quote}
Сума витрат на пальне: 65\euro{}
\end{quote}

\subsection{Зв'язок}

Без інтернету в Ісландії нікуди (інстаграмки там, фоточки, задовбування
фоловерів і т.д). Взяли стартовий пакет Simin --- із 1Гб включеного
трафіка та 100 хв. дзвінків. Ціна питання -- 20\euro{}

Декому до кінця подорожі трафіка не хватило, тому докупляли ще 500Мб ---
5\euro{}.

\begin{quote}
В загальному сума на зв'язок: 25\euro{}
\end{quote}

\subsection{Їжа \& алкоголь}


Частину їжі та алкоголю ми взяли із собою із України --- гречка,
вермішень, м'ясо, ковбасу, рис, мюслі, згущівку, і т.д. Цих харчів
хватило на 5 днів подорожі. Решту подорожі ми докупляли у місцевих
супермаркетах Bonus (одна із найдешевших мереж). Слід сказати, що навіть
найдешевші продукти дуже смачні в Ісландії.

Алкоголь в Ісландії (крім пива 2,5\%) продають лише у спеціальних
магазинах, які відчинені лише кілька годин і лише по буднях. Тому, якщо
плануєте багато випивати важкого алкоголю, краще його привезти з України


Ціна пива 2,5\% десь 0,8\euro{} за пляшку, звичайного пива --- десь
2-3\euro{}

\begin{quote}
Сума, яку потратили на їжу з людини -- 55\euro{}
\end{quote}

\subsection{Одяг і спорядження}

Намет + карімат

Побільше теплого одягу! Куртки, шапки, рукавиці, термобілизну --- беріть
все. Навіть в серпні у Ісландії холодно і вітер 

\subsection{Маршрут}

Маршрут ми планували на основі інформації, яку знайшли в неті. Вирішили
їхати за годинниковою стрілкою. Далі лінки на кожен із днів (буду
додавати лінки по мірі написання текстів).

День 0. Женева

День 1

День 2

День 3

День 4

День 5

День 6

День 7

День 8

День 9

День 10

День 11

День 12

День 13-14. Афтепаті

\subsection{Фінансові підсумки}

Вся поїздка (житло, літак, автомобіль, харчування, зв'язок) без
врахування особистих витрат: 705\euro{}

P.S. Далі спробую описувати побачене за кожен із днів і доповнювати цей
пост відповідними лінками. І якщо у вас є запитання, задавайте тут, у
коментах. Спробую відповідати

\href{http://dyoma.pp.ua/2015/09/02/dichslandiya-12-dniv-v-krajini-vodospadiv-hejzeriv-vulkaniv-ta-inshyh-vydumok-pryrody/}{Опубліковано
у блозі}
